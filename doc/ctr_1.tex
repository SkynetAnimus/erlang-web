\subsection{The simplest case}
The simplest scenario for running any dynamic content with ErlangWeb would be 
\begin{itemize}
  \item defining URL we want to access as static in the dispatcher (even if it is
    dynamic, it means that request go to template instead of going to controller at first)
  \item implement functionality in wpart\_\textit{new\_name}
  \item create template which uses wpart mentioned earlier
\end{itemize}
That is all. Although that is not real controller. It is possible to have any functionality in new wpart but in
complex projects it is not used often to display main content. Main reason is
that we want to parametrize template to behave different in many cases and it can be done in
controller before executing template. On the other hand, wpart in template
takes arguments - it is all right when they are simple, short and not dynamic
(e.g. customizable options in some select field or autocomplete field with
10000 choices). So when it is best to use static wparts? Lets take a look at \textit{top menu} on some website.
\begin{Verbatim}[numbers=left, fontsize=\small]
-module(wpart_menu).
-export([handle_call/1]).

-include_lib("xmerl/include/xmerl.hrl").

handle_call(_E) ->
    Options1 = [{"/", "HOME"},
		{"/news/", "NEWS"},
		{"/article/", "ARTICLES"},
		{"/blog/", "BLOGS"},
		{"/link/", "LINKS"},
		{"/doc.html", "DOCUMENTATION"}],
    
    LoggedRoot = filter:is_auth([root]),
    LoggedOther = filter:is_auth([blogger]),
    Options = if
		  LoggedRoot == true ->
		      Options1 ++ [{"/users/", "ADMIN "}];
		  LoggedOther == true ->
		      Options1 ++ [{"/logout/", "LOGOUT "}];
		  true ->
		      Options1
	      end,

    List = lists:map(fun({Link, Name}) ->
			    whelper:prepare("list_element", [Option, Name]]);
		     end, Options),
    Ready = whelper:prepare("list_cover", [List]]);			

    #xmlText{value=Ready, type=cdata}.
\end{Verbatim}
The example above is a simple version of menu. List of links is created. HTML code
is prepared by whelper [25]. Real menu also passes e.g. id = 'active' but for
simplicity it is cut off. Now, to use it, just type $<$wpart:menu /$>$ 
in proper place of base template.

\subsection{Basic Types}
\subsubsection{Definition options}
\begin{longtable}{|p{0.15\textwidth}|p{0.25\textwidth}|p{0.524\textwidth}|}
\hline
{\bf Type} & {\bf Argument} & {\bf Meaning} \\
\hline\hline
all types & \{private, Bool()\} & if Bool() is {\it true}, field is private - it will neither be checked during validation (always accepted) nor displayed in HTML form (even as hidden)

Example - field will be private:
\begin{verbatim}
{private, true}
\end{verbatim}\\ 
& & \\
& \{optional, DefaultValue\} & field will be optional - it will be set to its default value in case of not entering anything by user in HTML form

Example - field will be optional with default value set to integer 10:
\begin{verbatim}
{optional, 10}
\end{verbatim}\\
& & \\
& \{description, Desc\} & {\it Desc} will be the description displayed in the generated form. {\it Desc} could be either a string (and it will be a description itself) or tuple \{key, Key\} - in this situation description will be translated with the key {\it Key}

Example - field will be described as "Your name:":
\begin{verbatim}
{description, "Your name:"}
\end{verbatim}

Example - field will be described with translation of the key {\it user:login}:
\begin{verbatim}
{description, {key, "user:login"}}
\end{verbatim}\\
\hline
integer & \{min, Int()\} & checks if integer is greater or equal to the specified value. If not, {\it \{error, \{smaller\_than\_min, Int()\}\}} is returned

Example - integer must be greater than 2:
\begin{verbatim}
{min, 3}
\end{verbatim}\\
& \{max, Int()\} & checks if integer is smaller or equal to the specified value. If not, {\it \{error, \{greater\_than\_max, Int()\}\}} is returned

Example - integer must be smaller than 1000:
\begin{verbatim}
{max, 999}
\end{verbatim}\\ 
\hline
float & \{min, Float()\} & checks if float is greater or equal to the specified value. If not, {\it \{error, \{smaller\_than\_min, Float()\}\}} is returned

Example - float must be greater then 3.14:
\begin{verbatim}
{min, 3.14}
\end{verbatim}\\
& \{max, Float()\} & checks if float is smaller or equal to the specified value. If not, {\it \{error, \{greater\_than\_max, Float()\}\}} is returned

Example - float must be smaller than 256.23:
\begin{verbatim}
{max, 256.23}
\end{verbatim}\\ 
\hline
string, text & \{min\_length, Int()\} & checks if string is longer than or equal length as specified value. If not, {\it \{error, \{too\_short, Str()\}\}} is returned. Length of the string is measured by utf8\_api:ulength/1 function call

Example - string must be at least 6 characters long:
\begin{verbatim}
{min_length, 6}
\end{verbatim}\\
& \{max\_length, Int()\} & checks if string is shorter than or equal length as specified value. If not, {\it \{error, \{too\_long, Str()\}\}} is returned. Length of the string is measured by utf8\_api:ulength/1 function call

Example - string must be at most 20 characters long:
\begin{verbatim}
{max_length, 20}
\end{verbatim}\\
\hline
& \{html, Whitelist\} & check string for presence of valid XHTML tags. Only those tags, which are specified in whitelist are accepted.

The following errors ({\it \{error, \{not\_valid\_html, Reason\}\}} where Reason is one of the element below) are returned:
\begin{description}
\item[\{tags\_not\_closed, List\}]- user did not close all tags properly. Tags which are not closed are returned.
\item[\{tag\_not\_in\_whitelist, Tag\}]- user entered a tag, which is not in whitelist. Blacklisted tag is returned.
\item[\{closing\_bad\_tag, ClosingTag, OpenedTag\}] - user closed tag, which has not been opened most recently. Tag closed by user and the one, which should be closed are returned.
\item[\{no\_closing\_tag, Tag\}]- string has ended but tag remains opened (like "this is a link: <a").
\item[open\_tag\_inside\_tag]- user open tag inside another tag.
\item[\{no\_open\_quote, Tag\}]- user did not enquoted all attribute values properly.
\item[open\_tag\_inside\_attr]- user opened next tag inside of the attribute value.
\end{description}\\
& & Example - user is allowed to enter only {\it br}, {\it u}, {\it i} tags:
\begin{verbatim}
{html, ["br", "u", "i"]}
\end{verbatim}\\
\hline
date & \{format, Format\} & specifies the format of entered date. By default format is "YYYY-MM-DD". The accepted separators are "-", "/", " ", "." and "\_". If there is a bad separator entered, {\it \{error, \{bad\_separator\_in\_date\_form, Date\}\}} is returned. In case of entering bad input, {\it \{error, \{bad\_date\_format, Date\}\}} is returned.

Example - date must be in format "MM/DD/YYYY":
\begin{verbatim}
{format, "MM/DD/YYYY"}
\end{verbatim}\\
& \{min, MinDate\} & checks if entered date is later than specified. If not {\it \{error, \{bad\_range, Date\}\}} is returned.

Example - date must be later than 01/01/1970:
\begin{verbatim}
{min, "01/01/1970"}.
\end{verbatim}\\
& \{max, MaxDate\} & checks if entered date is earlier than specified. If not  {\it \{error, \{bad\_range, Date\}\}} is returned.

Example - date must be earlier than 06-12-2010:
\begin{verbatim}
{max, "06-12-2010"}
\end{verbatim}\\
\hline
password & \{min\_length,~Min\}, \{max\_length,~Max\} & the same as in {\it string, text}\\
\hline
atom & {\it none} & entered value atom representation must exist in the system\\
\hline
enum & \{choices, Choices\} & specifies the possible user can choose. {\it Choices} is string with elements separated with "\vline" character\\
\hline
bool & \{always, Bool()\} & checks if user's input is set to the given value. If not {\it \{error, \{bad\_bool\_value, Val\}\}} is returned.

Example - user have to enter false:
\begin{verbatim}
{always, false}
\end{verbatim}\\
\hline
multilist & \{options, Options\} & specifies the options which could be selected from list. {\it Options} is a string with elements separated with "\vline" character. Each element is of format value:description, e.g. "opt1:desc1\vline opt2:desc2".\\
\hline
upload & {\it none} & no additional options\\
\hline
time & \{format, Format\} & specifies the format of entered time. By default format is "HH:MM:SS". The only accepted separator is ":". If there is a bad separator entered, {\it \{error, \{bad\_separator\_in\_time\_form, Time\}\}} is returned. In case of entering bad input, {\it \{error, \{bad\_time\_format, Time\}\}} is returned.

Example - time must be in format "SS:MM:HH":
\begin{verbatim}
{format, "SS:MM:HH"}
\end{verbatim}\\
& \{min,~MinTime\}, \{max, MaxTime\} & the same as in {\it date}\\
\hline
autocomplete & \{min\_length,~Min\}, \{max\_length,~Max\}, \{html,~Whitelist\} & the same as in {\it string, text}\\
& & \\
& \{complete, List\} & lists the all possible autocompletion list. {\it List} should be a string with elements separated with~"\vline"\\
\hline
datetime &  \{format, Format\} & Example format "YYYY-MM-DD HH:MM:SS"  \\  
& \{max, MaxDateTime\} & \\ &  \{min, MinDateTime\} & Options are routed to
type date  and time validators -- so all acceptable date and time formats joined with
space are good as well \\
& & \\
& & Error codes: \emph{bad\_date}, \emph{bad\_time} -- \\
& & validation of time or
date failed; \emph{bad\_input} -- \\
& &  returned in case of wrongly formatted input \\
\hline
csv & \{type, Type\} & sets the type of the elements of entered data - all other options will be related to the pointed type. If there is an error in validation of the basic type element, {\it \{error, \{wrong\_value\_in\_set, CSV\}\}} is returned.

Example - csv of type integer:
\begin{verbatim}
{type, integer} 
\end{verbatim}\\
\hline
\end{longtable}

\clearpage
\subsubsection{HTML tags}
During HTML form build process, each element of the record (except private ones) is substituted with its corresponding HTML tag:
\begin{longtable}{|p{0.4\textwidth}|p{0.4\textwidth}|}
\hline
{\bf basic type} & {\bf HTML tag}\\
\hline\hline
integer & $<$input type="text" /$>$\\
\hline
float & $<$input type="text" /$>$\\
\hline
string & $<$input type="text" /$>$\\
\hline
date & $<$input type="text" /$>$\\
\hline
bool & $<$input type="checkbox" /$>$\\
\hline
enum & $<$input type="radio" /$>$\\
\hline
text & $<$textarea$>$...$<$/textarea$>$\\
\hline
upload & $<$input type="file" /$>$\\
\hline
password & $<$input type="password" /$>$\\
\hline
atom & $<$input type="text" /$>$\\
\hline
multilist & $<$select$>$\\
& $<$option$>$...$<$/option$>$\\
& ...\\
& $<$/select$>$\\
\hline
time & $<$input type="text" /$>$\\
\hline
datetime & $<$input type="text" /$>$\\
\hline
autocomplete & $<$input type="text" /$>$\\
& {\it (with JavaScript})\\
\hline
csv &  $<$input type="text" /$>$\\
\hline
\end{longtable}

\subsection{Pure dynamic}
Let's go step further. We don't want to play with HTML to build form and we
need some default values in displayed view. Moreover, it would be very helpful to get
error messages on bad values. Validation should be generic. To achieve that we will extend our
\emph{new\_contorller}. Example will present part of CRUD functionality (which
stands for {\bf C}reate {\bf R}ead {\bf U}pdate {\bf D}elete). Several prototypes of new projects in
ErlangWeb crafted optimal way of implementing it. 

\begin{Verbatim}[frame=single,
       framesep=2mm,
       label=dispacher.conf,labelposition=topline,
       fontsize=\small]
{dynamic, "^/before_create", {new_controller, before_create}}.
{dynamic, "^/create", {new_controller, create}}.
\end{Verbatim}
$ $
\begin{Verbatim}[frame=single,
       framesep=2mm,
       label=example\_records.hrl,labelposition=topline,
       fontsize=\small]
-record(example, 
	{name,
         city}
       ).

-record(example_types, 
	{name = {string, [{description, "Name"},
			   {min_length, 3}]},
	 city = {string, [{description, "City"},
                          {min_length, 5}]}}
       ).
\end{Verbatim}
$ $
\begin{Verbatim}[frame=single,
       framesep=2mm,
       label=save\_form.html,labelposition=topline,fontsize=\small]
...
 <wpart:lookup key="error_message" />
 <wpart:form type="example" action="\create"/> 
...
\end{Verbatim}
$ $
\begin{Verbatim}[frame=single,fontsize=\small,
       framesep=2mm,
       label=new\_controller.erl,labelposition=topline]
dataflow(before_create) -> [];
dataflow(create)       -> [validate];

% ..

validate(create,_) ->
	validate_tool:validate_cu(example,create);
% .. some code

%%
%% error handlers
%%
error(create, not_valid) ->
	Err = wpart:fget("__error"),
 	Message = "ERROR: Incomplete input or wrong 
                   type in form!"++ "  Reason: " ++ Err,
	wpart:fset("error_message",Message),

	Not_validated = wtype_users:prepare_initial(),
	wpart:fset("__edit", Not_validated),
	{template, "templates/users/create_users.html"};

% ..

%%
%% controller functions
%%
before_create(_) ->
            wpart:fset("error_message",""),
            {template, "templates/save_form.html"}.

create(X) -> 
            wtype_example:create(X),
            {redirect, "/save_form.html"}.
\end{Verbatim}

Ready! \emph{New\_controller} calls {\it wtype} but that's the subject of next
chapter. For now just imagine that it is there and works perfect. Function
\emph{before\_create} is responsible for displaying form. It uses
\emph{example\_records}. No HTML. Just define action under button in
template. Validation is on framework side. It returns proper values or
prepares error messages. In case of fail error/2 is called for
function which failed. In this case \emph{create}. On Error wrong fields
are surrounded by frame (table border) - style in css (class
{\it form\_error}). Furthermore, field in form has those wrong values, so it is
possible to make some changes. The same mechanism works for default
values. Just call from model prepare\_default instead of prepare\_initial. To
make it super simple function, create goes back to save form. Normally it would
be some menu template.
 

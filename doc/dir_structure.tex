\subsection{Overview}ErlangWeb is built upon OTP principles, which implies some conventions on it. One of them is directory structure. Project must contain at least the following set of directories:
\begin{itemize}
\item config
\item docroot
\item lib
\item log
\item pipes
\item priv
\item releases
\item templates
\end{itemize}

\subsection{Directory Meaning}
\begin{description}
\item[config] directory is the place where all configuration files should be placed. The must-have files are:
\begin{description}
\item[dispatch.conf]- routing rules for dispatcher
\item[errors.conf]- error templates definition
\item[project.conf]- main configuration file for the whole project
\item[configuration file for server]- server-specific configuration file (for example for Yaws it will be {\it yaws.conf})
\end{description}

Moreover, we should place here server certificates and keys files (for example in {\it config/keys}).

\item[docroot] directory is a folder for content, which should be served directly by server (without processing by our application), like CSS files, images, binary files, etc. To access it directly, we should place a static route rule in {\it dispatch.conf} file (or one of the included ones) with target set to {\bf enoent}.

\item[lib] directory is the place where applications created by users should be placed. Directories {\it lib/*/ebin} are automatically added to virtual machine path. After the installation, it should contain either symbolic links or copies of all used Erlang application, i.e. {\it stdlib}, {\it kernel}, {\it mnesia}, {\it compiler}, {\it eptic}, {\it wpart}, {\it wparts}, {\it runtime\_tools} and {\it yaws} or {\it inets}.

Each directory inside {\it lib} must satisfy OTP directory structure principles (contains at least folders: {\it src} for Erlang source code, {\it ebin} for Erlang object code and {\it .app} file, {\it priv} for application specific files and {\it include} for include files).

\item[log] directory is a container for logs created during running in embedded mode. Moreover, Yaws server default configuration file ({\it yaws.conf} copied from {\it Yaws} config directory) points this folder as a directory for its logs.

\item[pipes] directory holds the OS-level pipes for communicating with the shell of Erlang embedded system.

\item[priv] directory is a place for all project specific data: for example static html content (e.g. in {\it priv/static}).

\item[releases] directory contains separate subdirectories for each release version. Version folders should contain {\it .rel} file, boot script {\it start.boot} and optionally {\it relup} file. In the {\it releases} directory we have to put {\it sys.config} as well.

\item[templates] folder is a place for all template files used within the project. By default cache files are kept in {\it templates/cache} folder.
\end{description}

\clearpage
\subsection{Example of a directory tree}
\begin{verbatim}
.
 |..config
    |----dispatch.conf
    |----errors.conf
    |----project.conf
    |----yaws.conf
 |..docroot
 |..lib
    |----eptic
    .....|----ebin
    .....|----include
    .....|----priv
    .....|----src
%%...
    |----myapp
    .....|----ebin
    .....|----include
    .....|----priv
    .....|----src
    |----runtime_tools
    .....|----ebin
    .....|----include
    .....|----priv
    .....|----src
    |----second_app
    .....|----ebin
    .....|----include
    .....|----priv
    .....|----src
 |..log
 |..pipes
 |..priv
 |..releases
    |----sys.config
 |..templates
\end{verbatim}
